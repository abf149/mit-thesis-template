\chapter{SAF microarchitecture primitive taxonomy}
\label{chapter:primitive_taxo_model}

This section introduces a taxonomy of SAF microarchitecture primitives. Each taxonomic primitive expresses a basic operation that is key to implementing sparse tensor arithmetic, staying as close as possible to the \textit{Efficient Processing}\cite{szebook} taxonomy. 

The taxonomic system in this work is modeled off of C/C++ generics: 

\begin{itemize}
    \item 
\end{itemize}

Each taxonomic primitive has

\begin{itemize}
    \item \textbf{An interface}, comprising input and output ports. Each port has one of three \textit{Efficient Processing} interface types: \textit{metadata format}, \textit{position}, or \textit{flag}.
    \item \textbf{A set of attributes}, which serve to parameterize the primitive so that a wide range of implementation strategies may be encapsulated within the same abstract primitive but with different attribute value(s) for each implementation strategy.
    \item \textbf{An analytical energy/area/timing model}, which is parameterized by the primitive's attributes.
\end{itemize}

In order to stay as close to the \textit{Efficient Processing} taxonomy as possible, it is desirable that the primitive's input and output ports can be understood as moving metadata, position or flag values. Thus, SAF microarchitecture primitives are \textit{not} synonymous with RTL blocks from Chapter~\ref{chapter:rtl}, because RTL blocks are not \textit{necessarily} domain-specific in the way that SAF microarchitecture primitives are. A SAF microarchitecture primitive is constructed from one or more RTL blocks, 

Each of these primitives is associated with an analytical model, and larger SAF microarchitecture

In this work, SAF microarchitecture categorization is intended to facilitate communication of general design principles, very early planning discussions among designers, and design-space exploration of possible SAF microarchitectures. \todo{It is important to set up a systematic way of describing components and how they connect to each other. Later, we use this taxonomy as the basis of a set of tools for inferring the SAF microarchitecture design and modeling its energy, area and latency.} \\
%
A sufficiently expressive taxonomic framework encompasses SAF microarchitecture designs seen in the literature and may also express design not yet developed, enabling improvements on existing work. However a highly detailed and expressive representation of microarchitecture as, i.e., RTL, is not ideal because the design-space is huge and difficult to describe or constrain, or even to extract general design principles from. Existing architectural modeling tools face similar problems with exponential growth in the space of possible designs as the language for describing designs becomes more expressive, which they address by (1) limiting to a finite set of architectural primitives (i.e. SRAM, MAC, etc.)\cite{timeloop}, (2) supporting hierarchical component representations built out of primitives \cite{accelergy} such that a small primitive library enables a large space of designs, (3) parameterizing component categories with a concise set of attributes that are sufficient to narrow down an implementation \cite{accelergy}, (4) describing optimizations (i.e. SAFs as described in \cite{sparseloop}) as annotations on architectural components which themselves have attributes sufficient to describe the optimization strategy, and (5) producing frameworks which are extensible to help designers to concisely express key design-specific components. \todo{Antipattern - paper on specific microarchitectures only.} \\
%
This work addresses the challenge of describing SAF microarchitectures in an expressive but concise way, which is easily extensible by users. Ideally a designer could express many useful SAF microarchitectures as a netlist or schematic, not of logic gates, but of connections between higher-level ``SAF primitive components'' which recur frequently across published SAF microarchitecture designs. Components are further parameterized with attributes sufficient to narrow down the general energy, area and latency scaling behavior of the underlying implementation. To that end, the conceptual framework developed in this work consists of
%
\begin{itemize}
    \item A taxonomic categorization of SAF primitives
    \item For each taxonomic primitive, a set of attributes sufficient to narrow down the general type of microarchitecture implementation as well as the scaling behavior of energy, area and latency
    \item For each SAF described in \cite{sparseloop} (format, gating, skipping), a feasible set possible topologies for building the SAF microarchitecture out of SAF primitives
    \item For each SAF, a set of attributes which are sufficient to narrow down the appropriate SAF microarchitecture out of the feasible set. This set includes at least the attributes of the SAF itself (i.e. type of action optimization, or choice of representation format), as well as other attributes specific to the SAF microarchitecture.
\end{itemize}

This section will overview the taxonomy of microarchitectural primitive components considered in this work.

\section{Metadata parser taxonomic category template}

\begin{figure}[H]
    \centering
    \includegraphics[width=0.95\textwidth]{figures/mdparser.png}
    \caption{Metadata parser primitive component template.}
    \label{fig:mdparser}
\end{figure}

\begin{table}[H]
\centering
\begin{tabular}{lll}
\toprule
 format   & strategy       & coordinate\_arithmetic   \\
\midrule
 U        & sentinel & no\_arithmetic                \\
 U        & sentinel & yes\_arithmetic                 \\
 U        & length\_field & no\_arithmetic              \\
 U        & length\_field & yes\_arithmetic             \\
 C        & sentinel & no\_arithmetic                \\
 C        & sentinel & yes\_arithmetic                 \\
 C        & length\_field & no\_arithmetic              \\
 C        & length\_field & yes\_arithmetic             \\
 B        & sentinel & no\_arithmetic                  \\
 B        & sentinel & yes\_arithmetic                 \\
 B        & length\_field & no\_arithmetic              \\
 B        & length\_field & yes\_arithmetic             \\
\bottomrule
\end{tabular}
\caption{Specializations of metadata parsers.}
\label{tab:MetadataParser_specializations}
\end{table}

\subsection{Metadata parser models, by specialization}

\subsubsection{$(U,sentinel,*)$}

\subsubsection{$(U,length_field,*)$}

\subsubsection{$(C,sentinel,*)$}

\subsubsection{$(C,length_field,*)$}

\section{Position generators}

\subsection{Single position generator taxonomic category template}

\begin{figure}[H]
    \centering
    \includegraphics[width=0.95\textwidth]{figures/pgen1.png}
    \caption{Single position generator primitive component template.}
    \label{fig:pgen1}
\end{figure}

\begin{table}[H]
\centering
\begin{tabular}{lll}
\toprule
 format\_src   & format\_dst   & strategy    \\
\midrule
 C            & U            & passthrough \\
 C            & C            & counter     \\
\bottomrule
\end{tabular}
\caption{Specializations of single position-generator}
\label{tab:SinglePositionGenerator_specializations}
\end{table}

\subsubsection{Single position generator models, by specialization}

\subsection{Dual position generator taxonomic category template}

\begin{figure}[H]
    \centering
    \includegraphics[width=0.95\textwidth]{figures/pgen2.png}
    \caption{Dual position generator primitive component template.}
    \label{fig:pgen2}
\end{figure}

\begin{table}[H]
\centering
\begin{tabular}{llll}
\toprule
 format\_src   & format\_dst   & source\_strategy        & reference\_strategy             \\
\midrule
 B            & B            & ripple\_prefix\_sum      & parallel\_dec2\_priority\_encoder \\
 B            & B            & kogge\_stone\_prefix\_sum & parallel\_dec2\_priority\_encoder \\
\bottomrule
\end{tabular}
\caption{Specializations of dual position generator}
\label{tab:DualPositionGenerator_specializations}
\end{table}

\subsubsection{Dual position generator models, by specialization}

\section{Intersection units}

\subsection{Leader-follower intersection unit taxonomic category template}

\begin{figure}[H]
    \centering
    \includegraphics[width=0.95\textwidth]{figures/isectlf.png}
    \caption{Leader-follower intersection primitive component template.}
    \label{fig:isectlf}
\end{figure}

\begin{table}[H]
\centering
\begin{tabular}{ll}
\toprule
 format\_leader   & strategy    \\
\midrule
 C               & passthrough \\
 B               & passthrough \\
 R               & passthrough \\
\bottomrule
\end{tabular}
\caption{Specializations of leader-follower intersection.}
\label{tab:IntersectionLeaderFollower_specializations}
\end{table}

\subsubsection{Leader-follower intersection unit models, by specialization}

\subsection{Bidirectional intersection unit taxonomic category template}

\begin{figure}[H]
    \centering
    \includegraphics[width=0.95\textwidth]{figures/isectbd.png}
    \caption{Bidirectional intersection primitive component template.}
    \label{fig:isectbd}
\end{figure}

\begin{table}[H]
\centering
\begin{tabular}{lll}
\toprule
 format\_0   & format\_1   & strategy         \\
\midrule
 C          & C          & two\_finger\_merge \\
 C          & C          & skip\_ahead       \\
 C          & C          & direct\_map       \\
 B          & B          & bitwise\_and      \\
\bottomrule
\end{tabular}
\caption{Specializations of bidirectional intersection.}
\label{tab:IntersectionBidirectional_specializations}
\end{table}

\subsubsection{Bidirectional intersection unit models, by specialization}

\section{Fill optimizer taxonomic category template}

\begin{figure}[H]
    \centering
    \includegraphics[width=0.95\textwidth]{figures/fopt.png}
    \caption{Fill optimizer primitive component template. The fill optimizer's (fopt's) method for discarding payload fills - as well as the interface between the fopt and the payload stream - is highly implementation-dependent, thus the fopt template does not have a ``payload'' interface. The red dotted line from fopt to payload stream is a visual cue for which payload stream is being optimized, but in actuality \textit{pos\_in} is the only interface port defined for this primitive component template.}
    \label{fig:fopt}
\end{figure}

\begin{table}[H]
\centering
\begin{tabular}{l}
\toprule
 strategy             \\
\midrule
 pipeline\_bubble\_gate \\
 lut\_skip             \\
\bottomrule
\end{tabular}
\caption{Specializations of fill optimizer}
\label{tab:FillOptimizer_specializations}
\end{table}

\subsection{Fill optimizer models, by specialization}