\chapter{Explanation of PE area breakdowns from existing sparse tensor accelerator designs}

This section justifies the breakdowns of PE area for published sparse tensor accelerator designs, shown in Figure~\ref{fig:saf_uarch_significance}.

\section{GAMMA\cite{gamma}}

\section{SparTen\cite{sparten}}

\section{Eyeriss v2\cite{eyerissv2}}

This section justifies the claim in Figure~\ref{fig:saf_uarch_significance} that SAF microarchitecture comprises 7\% of Eyeriss v2 PE area. This claim is not made explicitly in the Eyeriss v2 paper\cite{eyerissv2}; rather, 7\% is an estimate I developed based on metrics that were provided in the paper.

The Eyeriss v2 PE loads activations from the weight scratchpad (weight spad), conditional on the compressed-sparse-column (CSC) metadata of input activation (iact) data being read from the iact spad\cite{eyerissv2}. This pattern of memory accesses can be understood as an $iact \rightarrow weight$ leader/follower skipping SAF. The use of CSC format for weights and iacts can be understood as a $T-<UC>(WH)$ format SAF. \textbf{Note that the use of SAF terminology and sparse tensor notation here is my own interpretation of Eyeriss v2, based on the definitions developed in \cite{sparseloop}\cite{szebook}.}

The leader/follower skipping SAF is lightweight: abstractly, the leader/follower intersection is simply a wire from iact data spad read output to iact address spad read input. Thus for a very rough estimate, the leader/follower intersection area overhead could be quantified as approximately zero. However, in practice there is control logic associated with this abstract wire which adds additional area to the skipping microarchitecture. 

Furthermore, the Eyeriss v2 PE maintains separate address and data memories for each datatype, and data memory is accessed indirectly by first doing a lookup into address memory\cite{eyerissv2}. 

Figure 17 of the Eyeriss v2 paper shows that the PE implements a pipeline for all of these layers of indirect memory access\cite{eyerissv2}. Thus, it is arguable that part of the purpose of the Eyeriss v2 PE pipeline is to implement the leader/follower skipping SAF and the CSC format SAF. For this reason, Figure~\ref{fig:saf_uarch_significance} counts a portion of the PE control logic when estimating the proportion of SAF microarchitecture area in the Eyeriss v2 PE (the iact and weight address buffers are arguably \textit{architectural} overheads required in the implementation of the CSC format SAF, and thus do not count towards SAF \textit{microarchitecture.})



\begin{table}
\caption{Eyeriss v2 PE area breakdown (Fig. 18b from \cite{eyerissv2})}
\label{tab:eyerissv2_pe_breakdown}
\label{arm:table}
\begin{center}
\begin{tabular}{||l|l|l||}\hline
\textbf{PE subcomponent(s)} & \textbf{Area} \%  & \textbf{Categorization based on this work}  \\\hline
spad	   & 74.4\% & Architectural memory \\\hline
- \textit{weight data}	   & \textit{21.5\%} &\\\hline
- \textit{weight addr}	   & \textit{5.7\%} & Architectural Format SAF overhead\\\hline
- \textit{iact data}	   & \textit{11.5\%} &\\\hline
- \textit{iact addr}	   & \textit{2.1\%} & Architectural Format SAF overhead \\\hline
- \textit{psum data}   & \textit{33.6\%} &\\\hline
compute (MAC)	   & 5.2\% & Architectural compute \\\hline
I/O FIFOs	   & 9.4\% & Non-SAF-related $\mu$arch\\\hline
control logic	   & 11.1\% & \textbf{Includes SAF $\mu$arch} \\\hline
\textbf{Total}	   & 100.1\% & \\\hline
\end{tabular}
\end{center}
\end{table}

Fig. 18b of the Eyeriss v2 paper\cite{eyerissv2} provides a breakdown of PE area, which is reproduced here in Table~\ref{tab:eyerissv2_pe_breakdown} (\textbf{Note that the categories assigned in the third column are my own interpretation and do not come from the Eyeriss v2 paper}.)

Figure~\ref{fig:saf_uarch_significance} does not count the full 11.1\% of control logic area (shown in Table~\ref{tab:eyerissv2_pe_breakdown}) as SAF microarchitecture. The estimate that SAF microarchitecture accounts for 7\% of PE area is based on examining the PE pipeline and developing a heuristic for the proportion of control logic which is attributable to SAF microarchitecture. The PE pipeline in Fig. 18b of the Eyeriss v2 paper\cite{eyerissv2} shows a five-stage pipeline (ignoring the single architectural register which holds an iact value stationary), and three of these pipeline stages arguably serve to implement SAFs: 

\begin{itemize}
    \item There is a set of pipeline registers between the weight address and weight data spads, and another set of registers between the iact address and iact data spads; since these registers pipeline the process of performing indirect lookups into data memory via lookups into address memory, \textit{here I categorize these two sets of registers as comprising parts of the format microarchitectures for the iact and weight spads.}
    
    \item There is a stage of pipeline registers after the iact spads and before the weight spads; this stage effectively pipelines the process of looking up weights based on iact metadata (i.e. leader/follower skipping), and \textit{therefore I categorize this pipeline stage as being part of the leader/follower skipping microarchitecture.}
\end{itemize}

Fig. 17 from the Eyeriss v2 paper visually suggests that all five pipeline stages are similar in size. Thus, about $\frac{3}{5}$ or 60\% of Eyeriss v2 pipeline area could potentially be categorized as SAF microarchitecture. Extrapolating this 60\% estimate from the pipeline area to the overall control logic area leads to the following estimate for the proportion of SAF microarchitecture area in the PE:

\[(11.1\%\;ctrl\,logic\,area)\times(60\%\;SAF\,\mu arch\,logic)\]

\[ = 6.9\%\; SAF\,\mu arch\,area\]


which is the source of the 7\% claim in Figure~\ref{fig:saf_uarch_significance}. 
\clearpage
\newpage
