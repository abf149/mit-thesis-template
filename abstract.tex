% $Log: abstract.tex,v $
% Revision 1.1  93/05/14  14:56:25  starflt
% Initial revision
% 
% Revision 1.1  90/05/04  10:41:01  lwvanels
% Initial revision
% 
%
%% The text of your abstract and nothing else (other than comments) goes here.
%% It will be single-spaced and the rest of the text that is supposed to go on
%% the abstract page will be generated by the abstractpage environment.  This
%% file should be \input (not \include 'd) from cover.tex.

% , utilizing simple formulae for \textit{architectural} compute and memory savings from sparse acceleration features (SAFs.) 

%Specialized microarchitectures for exploiting sparsity have been critical to the design of sparse tensor accelerators. Despite considerable research having been done, prior work lacks consistent abstractions, systematic design-space exploration, and apples-to-apples comparison between alternative microarchitecture proposals.

Specialized microarchitectures for exploiting sparsity have been critical to the design of sparse tensor accelerators. Sparseloop introduced the Sparse Acceleration Feature (SAF) abstraction, which unifies prior work on sparse tensor accelerators into a taxonomy of sparsity optimizations.

Sparseloop succeeds at analytical pre-RTL modeling of architecture-level metrics for sparse tensor accelerators, accurately capturing the beneficial impact of SAFs on overall design cost. However, Sparseloop lacks cost models for microarchitectural primitives and design topologies \textit{required for implementing SAFs} (referred to in this work as ``SAF microarchitectures''.) 

Analysis of prior works shows that SAF microarchitectures \textit{may or may not} constitute a significant overhead, depending on the particular design; thus it is desirable to have pre-RTL models which help anticipate SAF microarchitecture overheads.

Building on the Sparseloop SAF abstraction, this work\footnote{The repository for source code may be found at \url{https://github.com/abf149/saftool} (DOI: 10.5281/zenodo.10568496). This work has an MIT License.}
 attempts to synthesize a number of prior works into a concise, unified, and effective framework for doing research on SAF microarchitectures. This overall framework comprises (1) a conceptual framework which facilitates concise description and design-space exploration for SAF microarchitectures, (2) a software framework for compiling Sparseloop-style SAF descriptions into microarchitecture designs and analytical models, and (3) a component library including specific SAF microarchitecture subcomponent designs as well as RTL to support implementation.