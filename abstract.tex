% $Log: abstract.tex,v $
% Revision 1.1  93/05/14  14:56:25  starflt
% Initial revision
% 
% Revision 1.1  90/05/04  10:41:01  lwvanels
% Initial revision
% 
%
%% The text of your abstract and nothing else (other than comments) goes here.
%% It will be single-spaced and the rest of the text that is supposed to go on
%% the abstract page will be generated by the abstractpage environment.  This
%% file should be \input (not \include 'd) from cover.tex.
Researchers leverage existing fast analytical modeling frameworks for design-space exploration of sparse tensor accelerator architectures, utilizing simple formulae for \textit{architectural} compute and memory savings from sparse acceleration features (SAFs.) However, the general consensus on key architectural building blocks (ALU, cache, etc.), has no parallel at microarchitectural level, and thus no analytical models exist of SAFs' \textit{microarchitectural implementation overheads}. 

This paper presents a general approach to categorizing SAF microarchitectures, through which a number of prior works are synthesized into a concise, unified taxonomy of microarchitectures for the most common SAFs (format, gating and skipping.) Based on this taxonomy, SAFTools is a new software framework that helps researchers describe, model and implement SAF microarchitectures. SAFTools can consume a sparse architecture specification and (1) automatically design microarchiture(s) for all SAF(s) exploited by the architecture, (2) generate analytical models of SAF microarchitecture energy, latency and area, and (3) generate parameterized HDL for the SAF microarchitecture, to support implementation.

Extending an architectural analytical model framework for sparse accelerators with SAFTools-generated analytical models results in an $X\%$ modeling accuracy improvement against architectural modeling only. This work also presents case studies of design insights from SAF microarchitecture design-space exploration, which was previously impractical.
