% $Log: abstract.tex,v $
% Revision 1.1  93/05/14  14:56:25  starflt
% Initial revision
% 
% Revision 1.1  90/05/04  10:41:01  lwvanels
% Initial revision
% 
%
%% The text of your abstract and nothing else (other than comments) goes here.
%% It will be single-spaced and the rest of the text that is supposed to go on
%% the abstract page will be generated by the abstractpage environment.  This
%% file should be \input (not \include 'd) from cover.tex.

% , utilizing simple formulae for \textit{architectural} compute and memory savings from sparse acceleration features (SAFs.) 

Specialized microarchitectures for exploiting sparsity have critical to the design of sparse tensor accelerators. Despite considerable research having been done, prior work lacks consistent abstractions, systematic design-space exploration, and apples-to-apples comparison between alternative microarchitecture proposals.

Sparseloop (in combination with the Accelergy modeling framework) is an example of an existing tool which attempts to analytically model sparsity optimizations, known as Sparse Acceleration Features (SAFs.) Sparseloop effectively models key metrics such as energy, area and cycles at the architecture-level in the presence of sparsity optimizations. However, lacking models of microarchitectural primitives and design topologies, Sparseloop is unable to lower its sparsity abstractions onto a model of microarchitectural costs. Thus Sparseloop cannot capture the energy/area/time overhead incurred by SAF microarchitectures.

This work attempts to synthesize a number of prior works into a concise, unified, and effective framework for doing research on SAF microarchitectures. This overall framework comprises (1) a conceptual framework which facilitates concise description and design-space exploration for SAF microarchitectures, (2) a software framework for compiling Sparseloop-style SAF descriptions into microarchitecture designs and analytical models, and (3) a component library including specific SAF microarchitecture subcomponent designs as well as RTL to support implementation. 