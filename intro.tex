%% This is an example first chapter.  You should put chapter/appendix that you
%% write into a separate file, and add a line \include{yourfilename} to
%% main.tex, where `yourfilename.tex' is the name of the chapter/appendix file.
%% You can process specific files by typing their names in at the 
%% \files=
%% prompt when you run the file main.tex through LaTeX.
\chapter{Introduction}

Deep neural network (DNN)-based machine learning inference, scientific computing, graph analytics, and a variety of other disciplines rely heavily on tensor arithmetic primitives. Along with the demise of Moore's Law, this motivated the first generation of energy-efficient tensor accelerators~\cite{eyeriss}. More recently, a new generation of hardware accelerators are exploiting sparsity to improve efficiency~\cite{ampere}\cite{eyerissv2}\cite{sparten}\cite{sparch}\cite{scnn}\cite{candles}\cite{extensor}. In a significant amount of published sparse accelerator research, there is a lack of systematic process for designing the auxiliary microarchitecture that actually saves energy and cycles by exploiting sparsity. Authors design auxiliary microarchitectural components for exploiting sparsity (hereafter ``SAF microarchitectures'') through a process of either ad-hoc experimentation, or of constructing SAF microarchitecture to be compatible with other design choices, such as

\begin{itemize}

\item Architecture (i.e. MatRaptor's integration of sparse intersection into a systolic array\cite{matraptor})
\item Dataflow (i.e. GAMMA's microarchitecture for exploiting sparsity within a Gustavson dataflow\cite{gamma})
\item Other sparsity optimizations (i.e. SparTen's intersection unit which complies with the bitmask representation employed in the broader architecture\cite{sparten})
\item The structure of the tensor problem itself (i.e. the allusion to accelerating multi-operand tensor computations in ExTensor\cite{extensor}.) 
\end{itemize}

Experimenting with and optimizing SAF microarchitecture designs is valuable, and the SAF microarchitecture must be compatible with the broader design context. However, in all of these works there is a missed opportunity to define a SAF microarchitecture design space and explore it systematically in order to determine the optimal component design. This situation is further complicated by the lack of consistent abstractions for SAF microarchitecture, and the lack of support for SAF microarchitecture in existing pre-RTL modeling tools for estimating power, area and time (PAT)\cite{accelergy}. 

To that end, the objectives of this work are to (1) help designers describe, compare and explore sparse tensor accelerators, by proposing a systematic classification of SAF microarchitecture (in the form of a taxonomy), (2) support modeling SAF microarchitecture, by providing an open-source library of pre-RTL SAF microarchitecture primitive models based on consistent abstractions, and (3) support implementing SAF microarchitectures, by providing RTL reference designs for select SAF microarchitecture primitives. The remainder of Section~\ref{sec:intro} reviews background, Section~\ref{sec:related} gives an overview of related work, Section~\ref{sec:proposed} details the proposed thesis work and prioritization, Section~\ref{sec:methods} details methods, and Section~\ref{sec:timeline} details the timeline.

\begin{table*}[ht]
\begin{tabular}{c|p{2.5cm}|p{2.5cm}|p{2.5cm}|p{2.5cm}}
 & Modeling speed & Accurate SAF $\mu$architecture modeling & Consistent SAF $\mu$architecture abstractions? & Open-source SAF $\mu$architecture RTL?\\ \hline \hline
Architectural modeling frameworks & \textcolor{green}{\textbf{Fast}} & \textcolor{red}{\textbf{No}} & \textcolor{red}{\textbf{No}} & \textcolor{red}{\textbf{No}} \\ \hline
Design-specific models & \textcolor{red}{\textbf{Slow}} & \textcolor{green}{\textbf{Yes}} & \textcolor{red}{\textbf{No}} & \textcolor{red}{\textbf{Limited}} \\ \hline
$\mu$architecture taxonomy papers & \textcolor{red}{\textbf{N/A}} & \textcolor{red}{\textbf{No}} & \textcolor{red}{\textbf{Limited}} & \textcolor{red}{\textbf{No}} \\ \hline
\textbf{This work} & \textcolor{green}{\textbf{Fast}} & \textcolor{green}{\textbf{Yes}} & \textcolor{green}{\textbf{Yes}} & \textcolor{green}{\textbf{Yes}} \\ \hline
\end{tabular}
\label{tab:thiswork}
\caption{SAFTools and the underlying SAF microarchitecture taxonomy enable fast, accurate SAF microarchitecture modeling based on a consistent set of abstractions.}
\centering
\end{table*}