% -*-latex-*-
% 
% For questions, comments, concerns or complaints:
% thesis@mit.edu
% 
%
% $Log: cover.tex,v $
% Revision 1.9  2019/08/06 14:18:15  cmalin
% Replaced sample content with non-specific text.
%
% Revision 1.8  2008/05/13 15:02:15  jdreed
% Degree month is June, not May.  Added note about prevdegrees.
% Arthur Smith's title updated
%
% Revision 1.7  2001/02/08 18:53:16  boojum
% changed some \newpages to \cleardoublepages
%
% Revision 1.6  1999/10/21 14:49:31  boojum
% changed comment referring to documentstyle
%
% Revision 1.5  1999/10/21 14:39:04  boojum
% *** empty log message ***
%
% Revision 1.4  1997/04/18  17:54:10  othomas
% added page numbers on abstract and cover, and made 1 abstract
% page the default rather than 2.  (anne hunter tells me this
% is the new institute standard.)
%
% Revision 1.4  1997/04/18  17:54:10  othomas
% added page numbers on abstract and cover, and made 1 abstract
% page the default rather than 2.  (anne hunter tells me this
% is the new institute standard.)
%
% Revision 1.3  93/05/17  17:06:29  starflt
% Added acknowledgements section (suggested by tompalka)
% 
% Revision 1.2  92/04/22  13:13:13  epeisach
% Fixes for 1991 course 6 requirements
% Phrase "and to grant others the right to do so" has been added to 
% permission clause
% Second copy of abstract is not counted as separate pages so numbering works
% out
% 
% Revision 1.1  92/04/22  13:08:20  epeisach

% NOTE:
% These templates make an effort to conform to the MIT Thesis specifications,
% however the specifications can change. We recommend that you verify the
% layout of your title page with your thesis advisor and/or the MIT 
% Libraries before printing your final copy.
\title{MIT Thesis Template in Overleaf}

\author{Tim Beaver}
% If you wish to list your previous degrees on the cover page, use the 
% previous degrees command:
%       \prevdegrees{A.A., Harvard University (1985)}
% You can use the \\ command to list multiple previous degrees
%       \prevdegrees{B.S., University of California (1978) \\
%                    S.M., Massachusetts Institute of Technology (1981)}
\department{Department of Electrical Engineering and Computer Science}

% If the thesis is for two degrees simultaneously, list them both
% separated by \and like this:
% \degree{Doctor of Philosophy \and Master of Science}
\degree{Bachelor of Science in Computer Science and Engineering}

% As of the 2007-08 academic year, valid degree months are September, 
% February, or June.  The default is June.
\degreemonth{June}
\degreeyear{1990}
\thesisdate{May 18, 1990}

%% By default, the thesis will be copyrighted to MIT.  If you need to copyright
%% the thesis to yourself, just specify the `vi' documentclass option.  If for
%% some reason you want to exactly specify the copyright notice text, you can
%% use the \copyrightnoticetext command.  
%\copyrightnoticetext{\copyright IBM, 1990.  Do not open till Xmas.}

% If there is more than one supervisor, use the \supervisor command
% once for each.
\supervisor{William J. Supervisor}{Associate Professor}

% This is the department committee chairman, not the thesis committee
% chairman.  You should replace this with your Department's Committee
% Chairman.
\chairman{Arthur C. Chairman}{Chairman, Department Committee on Graduate Theses}

% Make the titlepage based on the above information.  If you need
% something special and can't use the standard form, you can specify
% the exact text of the titlepage yourself.  Put it in a titlepage
% environment and leave blank lines where you want vertical space.
% The spaces will be adjusted to fill the entire page.  The dotted
% lines for the signatures are made with the \signature command.
\maketitle

% The abstractpage environment sets up everything on the page except
% the text itself.  The title and other header material are put at the
% top of the page, and the supervisors are listed at the bottom.  A
% new page is begun both before and after.  Of course, an abstract may
% be more than one page itself.  If you need more control over the
% format of the page, you can use the abstract environment, which puts
% the word "Abstract" at the beginning and single spaces its text.

%% You can either \input (*not* \include) your abstract file, or you can put
%% the text of the abstract directly between the \begin{abstractpage} and
%% \end{abstractpage} commands.

% First copy: start a new page, and save the page number.
\cleardoublepage
% Uncomment the next line if you do NOT want a page number on your
% abstract and acknowledgments pages.
% \pagestyle{empty}
\setcounter{savepage}{\thepage}
\begin{abstractpage}
% $Log: abstract.tex,v $
% Revision 1.1  93/05/14  14:56:25  starflt
% Initial revision
% 
% Revision 1.1  90/05/04  10:41:01  lwvanels
% Initial revision
% 
%
%% The text of your abstract and nothing else (other than comments) goes here.
%% It will be single-spaced and the rest of the text that is supposed to go on
%% the abstract page will be generated by the abstractpage environment.  This
%% file should be \input (not \include 'd) from cover.tex.

% , utilizing simple formulae for \textit{architectural} compute and memory savings from sparse acceleration features (SAFs.) 

Specialized microarchitectures for exploiting sparsity have critical to the design of sparse tensor accelerators. Despite considerable research having been done, prior work lacks consistent abstractions, systematic design-space exploration, and apples-to-apples comparison between alternative microarchitecture proposals.

Sparseloop (in combination with the Accelergy modeling framework) is an example of an existing tool which attempts to analytically model sparsity optimizations, known as Sparse Acceleration Features (SAFs.) Sparseloop effectively models key metrics such as energy, area and cycles at the architecture-level in the presence of sparsity optimizations. However, lacking models of microarchitectural primitives and design topologies, Sparseloop is unable to lower its sparsity abstractions onto a model of microarchitectural costs. Thus Sparseloop cannot capture the energy/area/time overhead incurred by SAF microarchitectures.

This work attempts to synthesize a number of prior works into a concise, unified, and effective framework for doing research on SAF microarchitectures. This overall framework comprises (1) a conceptual framework which facilitates concise description and design-space exploration for SAF microarchitectures, (2) a software framework for compiling Sparseloop-style SAF descriptions into microarchitecture designs and analytical models, and (3) a component library including specific SAF microarchitecture subcomponent designs as well as RTL to support implementation. 
\end{abstractpage}

% Additional copy: start a new page, and reset the page number.  This way,
% the second copy of the abstract is not counted as separate pages.
% Uncomment the next 6 lines if you need two copies of the abstract
% page.
% \setcounter{page}{\thesavepage}
% \begin{abstractpage}
% % $Log: abstract.tex,v $
% Revision 1.1  93/05/14  14:56:25  starflt
% Initial revision
% 
% Revision 1.1  90/05/04  10:41:01  lwvanels
% Initial revision
% 
%
%% The text of your abstract and nothing else (other than comments) goes here.
%% It will be single-spaced and the rest of the text that is supposed to go on
%% the abstract page will be generated by the abstractpage environment.  This
%% file should be \input (not \include 'd) from cover.tex.

% , utilizing simple formulae for \textit{architectural} compute and memory savings from sparse acceleration features (SAFs.) 

Specialized microarchitectures for exploiting sparsity have critical to the design of sparse tensor accelerators. Despite considerable research having been done, prior work lacks consistent abstractions, systematic design-space exploration, and apples-to-apples comparison between alternative microarchitecture proposals.

Sparseloop (in combination with the Accelergy modeling framework) is an example of an existing tool which attempts to analytically model sparsity optimizations, known as Sparse Acceleration Features (SAFs.) Sparseloop effectively models key metrics such as energy, area and cycles at the architecture-level in the presence of sparsity optimizations. However, lacking models of microarchitectural primitives and design topologies, Sparseloop is unable to lower its sparsity abstractions onto a model of microarchitectural costs. Thus Sparseloop cannot capture the energy/area/time overhead incurred by SAF microarchitectures.

This work attempts to synthesize a number of prior works into a concise, unified, and effective framework for doing research on SAF microarchitectures. This overall framework comprises (1) a conceptual framework which facilitates concise description and design-space exploration for SAF microarchitectures, (2) a software framework for compiling Sparseloop-style SAF descriptions into microarchitecture designs and analytical models, and (3) a component library including specific SAF microarchitecture subcomponent designs as well as RTL to support implementation. 
% \end{abstractpage}

\cleardoublepage

\section*{Acknowledgments}

This is the acknowledgements section. You should replace this with your
own acknowledgements.

%%%%%%%%%%%%%%%%%%%%%%%%%%%%%%%%%%%%%%%%%%%%%%%%%%%%%%%%%%%%%%%%%%%%%%
% -*-latex-*-
